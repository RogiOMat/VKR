\newpage
\begin{center}
\large\textbf{Минобрнауки России}

\large\textbf{Юго-Западный государственный университет}
\vskip 1em
\normalsize{Кафедра программной инженерии}
\vskip 1em
\ifВКР{
        \begin{flushright}
        \begin{tabular}{p{.4\textwidth}}
        \centrow УТВЕРЖДАЮ: \\
        \centrow Заведующий кафедрой \\
        \hrulefill \\
        \setarstrut{\footnotesize}
        \centrow\footnotesize{(подпись, инициалы, фамилия)}\\
        \restorearstrut
        «\underline{\hspace{1cm}}»
        \underline{\hspace{3cm}}
        20\underline{\hspace{1cm}} г.\\
        \end{tabular}
        \end{flushright}
        }\fi
\end{center}
\vspace{1em}
  \begin{center}
  \large
\ifВКР{
ЗАДАНИЕ НА ВЫПУСКНУЮ КВАЛИФИКАЦИОННУЮ РАБОТУ
  ПО ПРОГРАММЕ БАКАЛАВРИАТА}
  \else
ЗАДАНИЕ НА КУРСОВУЮ РАБОТУ (ПРОЕКТ)
\fi
\normalsize
  \end{center}
\vspace{1em}
{\parindent0pt
  Студента \АвторРод, шифр\ \Шифр, группа \Группа
  
1. Тема «\Тема\ \ТемаВтораяСтрока»
\ifВКР{
утверждена приказом ректора ЮЗГУ от \ДатаПриказа\ № \НомерПриказа
}\fi.

2. Срок предоставления работы к защите \СрокПредоставления

3. Исходные данные для создания программной системы:

3.1. Перечень решаемых задач:}

\renewcommand\labelenumi{\theenumi)}

\begin{enumerate}
	\item захват и воспроизведение аудио в реальном времени;
	\item загрузка и обработка аудиофайлов;
	\item разделение сигнала на частотные полосы;
	\item многополосная компрессия с индивидуальными настройками;
	\item эквализация для коррекции тембра;
	\item реверберация и шумоподавление;
	\item оптимизация производительности (JIT, многопоточность);
	\item управление состоянием и синхронизация модулей;
	\item настройка параметров эффектов в реальном времени;
	\item логирование и диагностика работы системы.
\end{enumerate}

{\parindent0pt
  3.2. Входные данные и требуемые результаты для программы:}

\begin{enumerate}
\item Входными данными для программной системы являются: аудиосигнал в виде цифрового потока, настройки эффектов, задаваемые пользователем (параметры эквалайзера, компрессора, реверберации и шумоподовление).
\item Выходными данными для программной системы являются: обработанный аудиосигнал, выводимый на устройство воспроизведения или сохраняемый в файл, возможность мониторинга и визуализации параметров обработки.
\end{enumerate}

{\parindent0pt

  4. Содержание работы (по разделам):
  
  4.1. Введение.
  
  4.1. Анализ предметной области.
  
4.2. Техническое задание: основание для разработки, цель и назначение разработки, требования к программной системе, требования пользователя к интерфейсу.

4.3. Технический проект: общая характеристика решения задачи, обоснования выбора технологии, эффекты, архитекрута программной системы.

4.4. Рабочий проект: спецификация классов программы, тестирование системы.

4.5. Заключение.

4.6. Список использованных источников.

5. Перечень графического материала:

\списокПлакатов

\vskip 2em
\begin{tabular}{p{6.8cm}C{3.8cm}C{4.8cm}}
Руководитель \ifВКР{ВКР}\else работы (проекта) \fi & \lhrulefill{\fill} & \fillcenter\Руководитель\\
\setarstrut{\footnotesize}
& \footnotesize{(подпись, дата)} & \footnotesize{(инициалы, фамилия)}\\
\restorearstrut
Задание принял к исполнению & \lhrulefill{\fill} & \fillcenter\Автор\\
\setarstrut{\footnotesize}
& \footnotesize{(подпись, дата)} & \footnotesize{(инициалы, фамилия)}\\
\restorearstrut
\end{tabular}
}

\renewcommand\labelenumi{\theenumi.}
