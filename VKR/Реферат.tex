\abstract{РЕФЕРАТ}

Объем работы равен \formbytotal{lastpage}{страниц}{е}{ам}{ам}. Работа содержит \formbytotal{figurecnt}{иллюстраци}{ю}{и}{й}, \formbytotal{tablecnt}{таблиц}{у}{ы}{}, \arabic{bibcount} библиографических источников и \formbytotal{числоПлакатов}{лист}{}{а}{ов} графического материала. Количество приложений – 2. Графический материал представлен в приложении А. Фрагменты исходного кода представлены в приложении Б.

Перечень ключевых слов: аудиопроцессор, цифровая обработка сигналов, Python, многополосная компрессия, эквализация, реверберация, шумоподавление, многопоточность, JIT-компиляция, оптимизация, архитектура программной системы, аудиофильтры, обработка в реальном времени.

Объектом разработки является программная система — универсальный аудиопроцессор, предназначенный для обработки аудиосигналов в реальном времени с использованием современных методов цифровой обработки сигналов.

Целью выпускной квалификационной работы является создание программного аудиопроцессора, обеспечивающего высокое качество обработки звука и возможность гибкой настройки параметров для различных пользовательских задач.

В процессе разработки аудиопроцессора были реализованы основные модули для захвата и воспроизведения аудиосигнала, разделения сигнала на частотные полосы, многополосной компрессии, эквализации, реверберации и шумоподавления. Для повышения производительности использованы технологии JIT-компиляции и многопоточности. Архитектура системы построена на принципах модульности и масштабируемости, что позволяет легко расширять функциональность и адаптировать продукт под различные сценарии использования.

Разработанный аудиопроцессор протестирован и готов к использованию для обработки аудиосигналов в реальном времени.

\selectlanguage{english}
\abstract{ABSTRACT}

The volume of work is \formbytotal{lastpage}{page}{}{s}{s}. The work contains \formbytotal{figurecnt}{illustration}{}{s}{s}, \formbytotal{tablecnt}{table}{}{s}{s}, \arabic{bibcount} bibliographic sources and \formbytotal{числоПлакатов}{sheet}{}{s}{s} of graphic material. The number of applications is 2. The graphic material is presented in annex A. The layout of the site, including the connection of components, is presented in annex B.

Keywords: audio processor, digital signal processing, Python, multiband compression, equalization, reverberation, noise reduction, multithreading, JIT compilation, optimization, software system architecture, audio filters, real-time processing.

The object of the development is a software system, a universal audio processor designed to process audio signals in real time using modern digital signal processing methods.

The purpose of the final qualification is to create a software audio processor that provides high-quality audio processing and the ability to flexibly adjust parameters for various user tasks.

During the development of the audio processor, the main modules for capturing and reproducing the audio signal, dividing the signal into frequency bands, multiband compression, equalization, reverberation and noise reduction were implemented. JIT compilation and multithreading technologies are used to improve performance. The architecture of the system is based on the principles of modularity and scalability, which makes it easy to expand functionality and adapt the product to various use cases.

The developed audio processor has been tested and is ready for use for processing audio signals in real time.

\selectlanguage{russian}
