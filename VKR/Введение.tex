\section*{ВВЕДЕНИЕ}
\addcontentsline{toc}{section}{ВВЕДЕНИЕ}

В современную эпоху цифровых технологий качество звука становится одним из ключевых факторов успешной коммуникации, развлечений и творчества. Развитие аудиотехнологий сопровождается постоянным ростом требований к обработке аудиосигналов как в профессиональной, так и в бытовой сфере. В связи с этим возрастает актуальность создания эффективных программных решений, позволяющих улучшать, корректировать и адаптировать звук под различные задачи и условия.

Программные аудиопроцессоры занимают важное место в индустрии обработки звука, обеспечивая широкий спектр возможностей: от базовой фильтрации и эквализации до сложной динамической обработки, реверберации и подавления шумов. Современные программные комплексы должны не только обеспечивать высокое качество обработки, но и быть гибкими, масштабируемыми, работать в реальном времени и иметь удобный интерфейс для пользователя.

Целью данной дипломной работы является разработка программного аудиопроцессора на языке Python, способного выполнять многополосную обработку аудиосигнала в реальном времени с применением современных эффектов, таких как эквализация, компрессия, реверберация и шумоподавление. Особое внимание уделяется модульности архитектуры, оптимизации производительности и возможности гибкой настройки параметров обработки.

\emph{Цель настоящей работы} - разработка аудиопроцессора с графическим интерфейсом для эффективной обработки звука с поддержкой основных эффектов. Для достижения поставленной цели необходимо решить следующие задачи:
\begin{itemize}
	\item провести анализ предметной области и существующих решений;
	\item спроектировать архитектуру программно-информационной системы, обеспечивающую модульность, расширяемость и эффективное взаимодействие всех компонентов;
	\item реализовать модуль обработки и воспроизведения аудиосигнала с поддержкой работы в реальном времени, реализовать основные эффекты обработки звука;
	\item создание удобного и понятного интерфейса приложение с отображение графиков формы волны и АЧХ в реальном времени и в соответствии с обработанными данными;
	\item обеспечить оптимизацию вычислений и минизацию задержек. 
\end{itemize}	

\emph{Структура и объем работы.} Отчет состоит из введения, 4 разделов основной части, заключения, списка использованных источников, 2 приложений. Текст выпускной квалификационной работы равен \formbytotal{lastpage}{страниц}{е}{ам}{ам}.

\emph{Во введении} сформулирована цель работы, поставлены задачи разработки, описана структура работы, приведено краткое содержание каждого из разделов.

\emph{В первом разделе} проведён анализ предметной области, включающий основные понятия цифровой обработки звука, историю развития аудиопроцессоров, их классификацию, технические проблемы и перспективы развития.

\emph{Во втором разделе} представлено техническое задание, включающее основания для разработки, цели, требования к системе, функциональные и нефункциональные требования, а также моделирование вариантов использования.

\emph{В третьем разделе} описан технический проект, включающий общую характеристику решения, используемые технологии, устройство эффектов (шумоподавление, эквалайзер, компрессор, реверберация), архитектуру системы и проектирование пользовательского интерфейса.

\emph{В четвертом разделе} приведён рабочий проект, содержащий описание классов приложения, элементов интерфейса и результаты тестирования программной системы.

В заключении излагаются основные результаты работы, полученные в ходе разработки.

В приложении А представлен графический материал.
В приложении Б представлены фрагменты исходного кода. 