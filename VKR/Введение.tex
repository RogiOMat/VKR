\section*{ВВЕДЕНИЕ}
\addcontentsline{toc}{section}{ВВЕДЕНИЕ}

В современную эпоху цифровых технологий качество звука становится одним из ключевых факторов успешной коммуникации, развлечений и творчества. Развитие аудиотехнологий сопровождается постоянным ростом требований к обработке аудиосигналов как в профессиональной, так и в бытовой сфере. В связи с этим возрастает актуальность создания эффективных программных решений, позволяющих улучшать, корректировать и адаптировать звук под различные задачи и условия.

Программные аудиопроцессоры занимают важное место в индустрии обработки звука, обеспечивая широкий спектр возможностей: от базовой фильтрации и эквализации до сложной динамической обработки, реверберации и подавления шумов. Современные программные комплексы должны не только обеспечивать высокое качество обработки, но и быть гибкими, масштабируемыми, работать в реальном времени и иметь удобный интерфейс для пользователя.

Целью данной дипломной работы является разработка программного аудиопроцессора на языке Python, способного выполнять многополосную обработку аудиосигнала в реальном времени с применением современных эффектов, таких как эквализация, компрессия, реверберация и шумоподавление. Особое внимание уделяется модульности архитектуры, оптимизации производительности и возможности гибкой настройки параметров обработки.

Для реализации поставленной цели были сформулированы следующие задачи:
\begin{itemize}
	\item разработка оптимальных алгоритмов реализации шумоподавления, эквализации, компрессии и реверберации;
	\item спроектировать архитектуру программной системы, обеспечивающую модульность, расширяемость и эффективное взаимодействие всех компонентов;
	\item реализовать модуль обработки и воспроизведения аудиосигнала с поддержкой работы в реальном времени;
	\item создание удобного и понятного интерфейса приложения с отображением графиков формы волны.
\end{itemize}	

В ходе работы рассматриваются вопросы цифровой обработки сигналов, проектирования программных систем, оптимизации вычислений с помощью современных библиотек Python, а также вопросы взаимодействия с аудиоустройствами. Практическая значимость разработки заключается в создании универсального инструмента, который может быть использован как в профессиональной деятельности звукорежиссёров и музыкантов, так и в бытовых условиях для улучшения качества звука при прослушивании, записи или трансляции аудио.

Таким образом, данная работа направлена на решение актуальных задач цифровой обработки аудиосигналов и создание эффективного, удобного и расширяемого программного аудиопроцессора, отвечающего современным требованиям к качеству звука.