\section*{ЗАКЛЮЧЕНИЕ}
\addcontentsline{toc}{section}{ЗАКЛЮЧЕНИЕ}

В ходе выполнения данной дипломной работы был разработа программный аудиопроцессор на языке Python, способный выполнять многополосную обработку аудиосигнала в реальном времени с применением современных эффектов, таких как компрессия, эквализация, реверберация и шумоподавление. В процессе работы был создан модуль архитектуры, произведена оптимизация производительности и реализована возможность гибкой настройки параметров обработки.

Задачи, поставленные в начале разработки были решены следующим образом:
\begin{itemize}
	\item провёл анализ предметной области и существующих решений;
	\item спроектировал архитектуру программно-информационной системы, обеспечивающую модульность, расширяемость и эффективное взаимодействие всех компонентов;
	\item реализовал модуль обработки и воспроизведения аудиосигнала с поддержкой работы в реальном времени, реализовал основные эффекты обработки звука;
	\item создал удобный и понятный интерфейс приложения с отображение графиков формы волны и АЧХ в реальном времени и в соответствии с обработанными данными;
	\item обеспечена оптимизацию вычислений и минизацию задержек.
\end{itemize}

Особое внимание уделялось оптимизации производительности с использованием технологий JIT-компиляции (Numba) и многопоточности, что позволило обеспечить обработку аудиосигнала в реальном времени с минимальной задержкой. Разработанный аудиопроцессор продемонстрировал высокое качество звуковой обработки, гибкость настройки параметров и устойчивость к ошибкам.

Результаты работы могут быть использованы как основа для дальнейшего развития программных аудиопроцессоров, расширения функционала и интеграции с другими аудиосистемами. Практическая значимость проекта заключается в создании универсального инструмента для динамической обработки звука, который может применяться в профессиональной звукозаписи, радиовещании, а также в бытовых аудиоприложениях.

Таким образом, поставленные цели и задачи дипломной работы полностью достигнуты, а разработанная система соответствует современным требованиям к качеству и эффективности цифровой обработки аудиосигналов.